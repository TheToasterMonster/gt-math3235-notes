\chapter{Limit Theorems}

\section{Convergence of Random Variables}

\begin{example}
  Let $\{a_n\}_{n = 1}^\infty$ be a sequence of
  real numbers such that $a_n \to 0$ as $n \to \infty$.
  Then
  \[
    \frac{a_1 + \dots + a_n}{n} \xrightarrow[n \to \infty]{} 0
  \]
  as well. This is known as \emph{Ces\`aro convergence}.
  We sometimes write $a_1 + \dots + a_n = \sum_{k = 1}^n a_k = s_n$.
\end{example}

\begin{remark}
  What does it mean for a sequence of
  random variables to converge? Let $X_1, \dots, X_n, \dots$
  be a sequence of random variables. Each
  $X_k$ is a function $X_k : \Omega \to \R$.
  One possible notion of convergence is
  the pointwise convergence of the $X_k$. But this is
  not very useful as it is not very
  likely. If the $X_k$ are
  coin flips, then we get a random
  sequence of $0$s and $1$s. This probably does
  not converge pointwise.
\end{remark}

\begin{definition}
  A sequence $\{X_n\}_{n = 1}^\infty$ of random
  variables \emph{converges in mean-square}
  to $X$ if
  \[
    \lim_{n \to \infty} \EE[(X_n - X)^2] = 0.
  \]
  Convergence in mean-square is often denoted
  by $X_n \xrightarrow[n \to \infty]{\mathrm{m.s.}} X$.
\end{definition}

\begin{definition}
  A sequence $\{X_n\}_{n = 1}^\infty$ of random
  variables \emph{converges in probability} to
  $X$ if for all $\epsilon > 0$,
  \[
    \PP(|X_n - X| \ge \epsilon) \xrightarrow[n \to \infty]{} 0.
  \]
  Convergence in probability is often denoted
  by $X_n \xrightarrow[n \to \infty]{\PP} X$.
\end{definition}

\begin{example}
  Define $X_n$ by
  $\PP(X_n = 0) = 1 - 1 / n$ and
  $\PP(X_n = n) = 1 / n$.
  In this case, we have $X_n \to 0$ in probability.
  To see this, fix $\epsilon > 0$, and we can compute
  that
  \[
    \PP(|X_n| \ge \epsilon)
    = \PP(X_n = n) = \frac{1}{n} \xrightarrow[n \to \infty]{} 0.
  \]
\end{example}

\begin{remark}
  The idea behind mean-square convergence is that
  we have
  $\EE[Y^2] = 0$ if and only if $Y = 0$ almost surely.
  One difference is that the $X_n$ must have
  finite second moments in order to talk about
  mean-square convergence, whereas this is
  not necessary for convergence in probability.
\end{remark}

\begin{example}
  Define $X_n$ the same way as in the previous
  example. We can compute that
  \[
    \EE[X_n]^2 = \frac{n^2}{n} = n,
  \]
  which does not converge to $0$ as $n \to \infty$.
  Thus $X_n$ does not converge to $0$ in mean-square.
  In particular, this means that convergence
  in probability does not imply convergence in
  mean-square.
\end{example}

\begin{example}
  Now define $X_n$ by
  $\PP(X_n = 1) = 1 / n$ and
  $\PP(X_n = 2) = 1 - 1 / n$. We can compute
  \[
    \EE[(X_n - 2)^2]
    = (1 - 2)^2 \frac{1}{n} + (2 - 2)^2 \left(1 - \frac{1}{n}\right)
    = \frac{1}{n} \xrightarrow[n \to \infty]{} 0,
  \]
  so we can conclude that
  $X_n \xrightarrow[n \to \infty]{\mathrm{m.s.}} 2$.
\end{example}

\begin{prop}
  Convergence in mean-square implies convergence
  in probability.
\end{prop}

\begin{proof}
  Recall that if $X \ge 0$ and $x > 0$, then by Markov's inequality
  we have
  \[
    \PP(X \ge x) \le \frac{\EE[X]}{x}.
  \]
  Applying this inequality to $|X_n - X|$ and $\epsilon > 0$
  gives us
  \[
    \PP(|X_n - X| \ge \epsilon)
    = \PP(|X_n - X|^2 \ge \epsilon^2)
    \le \frac{\EE[(X_n - X)^2]}{\epsilon^2}
    \xrightarrow[n \to \infty]{} 0
  \]
  when $X_n \xrightarrow[n \to \infty]{\mathrm{m.s.}} X$,
  since $\epsilon > 0$ is fixed. This shows
  convergence in probability.
\end{proof}

\section{Weak Law of Large Numbers}
\begin{theorem}[Weak law of large numbers]
  Let $X_1, X_2, \dots, X_n, \dots$ be a sequence
  of independent random variables, each with
  mean $\mu$ and variance $\sigma^2$. Then
  \[
    \frac{1}{n}(X_1 + \dots + X_n) \xrightarrow[n \to \infty]{\mathrm{m.s.}} \mu.
  \]
\end{theorem}

\begin{proof}
  Let $S_n = X_1 + \dots + X_n$.
  Since $\EE[X_1] = \EE[X_2] = \dots = \EE[X_n] = \mu$,
  we have
  \[
    \EE\left[\frac{S_n}{n}\right]
    = \frac{\mu_1 + \dots + \mu_n}{n} = \frac{n\mu}{n}
    = \mu.
  \]
  Now we can compute that
  \begin{align*}
    \EE\left[\left(\frac{S_n}{n} - \mu\right)^2\right]
    &= \EE\left[\left(\frac{S_n}{n} - \frac{\EE[S_n]}{n}\right)^2\right]
    = \frac{1}{n^2} \EE[(S_n - \EE[S_n])^2] \\
    &= \frac{1}{n^2} \Var[S_n]
    = \frac{1}{n^2} \sum_{k = 1}^n \Var[X_k]
    = \frac{n\sigma^2}{n^2} = \frac{\sigma^2}{n}
    \xrightarrow[n \to \infty]{} 0,
  \end{align*}
  where $\Var[S_n] = \sum_{k = 1}^n \Var[X_k]$ by independence.
  This implies that $S_n / n \xrightarrow[n \to \infty]{\mathrm{m.s.}} \mu$.
\end{proof}

\begin{remark}
  In the above proof, it is enough to assume that the
  $X_i$ are uncorrelated in place of independence.
  It is also enough to assume that the variances
  $\Var[X_i]$ are uniformly bounded, i.e.
  $\Var[X_i] \le M$ for every $i$. In this case,
  we can replace $\sigma^2$ with $M$ and still get
  $M / n \to 0$ as $n \to \infty$.
\end{remark}

\begin{remark}
  In ergodic theory, this type of result is known
  as an \emph{ergodic theorem}.
\end{remark}

\begin{example}
  Suppose that $X_i \sim \Ber(p)$ for $i \ge 1$
  are independent. Note that $\EE[X_i] = p$.
  Then the weak law of large numbers says that
  we have
  \[
    \frac{X_1 + \dots + X_n}{n} \xrightarrow[n \to \infty]{\mathrm{m.s.}} \EE[X_1] = p = \PP(X_1 = 1).
  \]
  Mean-square convergence then implies
  convergence in probability, so
  this says that averaging the flips of a coin
  will give the probability of getting heads
  as the number of flips $n \to \infty$.
\end{example}

\begin{remark}
  The (strong) law of large numbers says that
  we also get convergence \emph{almost surely}.
\end{remark}

\section{Central Limit Theorem}

\begin{remark}
  The weak law of large numbers says that
  $S_n / n \approx \mu$, i.e.
  $S_n \approx n \mu$. But this is a first order
  approximation: $n \mu \in \R$ is a constant,
  but $S_n$ is random. Observe that
  $S_n - n\mu$ is random and
  \[
    \EE[S_n - n\mu] = 0
  \]
  if all of the $X_k$ have the same mean.
  This means that $S_n - n\mu$ has
  mean $0$ and variance
  \[
    \Var[S_n - n\mu]
    = \Var[S_n] = n\sigma^2
  \]
  if all of the $X_k$ have the same variance.
  In particular, these calculations imply that
  \[
    Z_n = \frac{S_n - n\mu}{\sqrt{n\sigma^2}}
  \]
  has mean $0$ and variance $1$. The central
  limit theorem says that in fact
  $Z_n \approx \mathcal{N}(0, 1)$.
\end{remark}

\begin{remark}
  The convergence of $Z_n$ to
  $\mathcal{N}(0, 1)$ is not in mean-square,
  in probability, or almost surely.
\end{remark}

\begin{theorem}[Central limit theorem]
  Let $X_1, X_2, \dots, X_n, \dots$ be i.i.d.
  random variables with mean $\mu$ and
  variance $\sigma^2 > 0$. Then
  \[
    Z_n = \frac{S_n - n\mu}{\sqrt{n\sigma^2}}
  \]
  converges to $\mathcal{N}(0, 1)$ in distribution,
  i.e.
  $F_{Z_n}(x) \to \Phi(x)$
  pointwise as $n \to \infty$.
\end{theorem}

\begin{proof}
  Assume each $X_k$ has a finite mgf, i.e. $M_{X_k}(t) = \EE[e^{tX_k}]$
  exists for all $t \in \R$. Then
  \[
    M_{Z_n}(t)
    = \EE\left[\exp\left(t \cdot \frac{S_n - n\mu}{\sqrt{n\sigma^2}}\right)\right]
    = \EE\left[\exp\left(\frac{t}{\sqrt{n \sigma^2}} (S_n - n\mu)\right)\right].
  \]
  Since the $X_k$ are i.i.d., we obtain
  \[
    M_{Z_n}(t) = \prod_{k = 1}^n \EE\left[\exp\left(\frac{t}{\sqrt{n \sigma^2}} (X_k - \mu)\right)\right]
    = \left(\EE\left[\exp\left(\frac{t}{\sqrt{n \sigma^2}} (X_1 - \mu)\right)\right]\right)^n.
  \]
  Using the power series expansion for
  $\exp(x)$, this becomes
  \[
    M_{Z_n}(t)
    = \left(\EE\left[1 + \frac{t}{\sqrt{n\sigma^2}}(X_1 - \mu) + \frac{t^2}{2 n \sigma^2}(X_1 - \mu)^2 + \frac{t^3}{6 (n\sigma^2)^{3 / 2}} (X_1 - \mu)^3 + \cdots\right]\right)^n.
  \]
  By the uniform convergence of the
  power series for $e^x$ and using
  $\EE[X_1 - \mu] = 0$, we obtain
  \begin{align*}
    M_{Z_n}(t)
    &= \left(1 + \frac{t}{\sqrt{n\sigma^2}} \EE[X_1 - \mu] + \frac{t^2}{2n\sigma^2} \EE[(X_1 - \mu)^2] + o(1 / n)\right)^n \\
    &= \left(1 + \frac{t^2}{2n} + o(1 / n)\right)^n
    \xrightarrow[n \to \infty]{} e^{t^2 / 2}
    = M_{\mathcal{N}(0, 1)}(t),
  \end{align*}
  where we recalled that
  $(1 + x / n)^n \to e^x$ as $n \to \infty$.
  Applying the inversion formula for mgfs, one
  can obtain that $F_{Z_n}(x) \to \Phi(x)$
  as $n \to \infty$. By using the characteristic
  function in place of the mgf when it does not
  exist and carrying through
  a similar argument, one obtains the result in
  general.
\end{proof}

\begin{remark}
  Another way to think of the central limit theorem
  is the following: If $X_1, \dots, X_n$
  are independent random variables with
  finite second moment, i.e.
  $\EE[X_k^2] < \infty$ for each $1 \le k \le n$.
  Then
  \[
    S_n = X_1 + \dots + X_n
  \]
  is approximately normal, i.e.
  $F_{S_n}$ is approximately the cdf of a normal
  random variable with
  \[
    \mu = \sum_{k = 1}^n \EE[X_k]
    \quad \text{and} \quad
    \sigma^2 = \sum_{k = 1}^n \Var[X_k].
  \]
  In particular, if we recenter and rescale
  $S_n$, then we get that
  \[
    \frac{(X_1 + \dots + X_n) - \sum_{k = 1}^n \EE[X_k]}{\sqrt{\sum_{k = 1}^n \Var[X_k]}}
    \approx \mathcal{N}(0, 1).
  \]
  In even more generality, if
  $a_1, \dots, a_n \in \R$, then
  \[
    L_n = \sum_{k = 1}^n a_k X_k
    \approx \mathcal{N}\left(\sum_{k = 1}^n a_k \EE[X_k], \sum_{k = 1}^n a_k^2 \Var[X_k]\right).
  \]
  The appropriate recentering and rescaling can be
  done to relate this again to $\mathcal{N}(0, 1)$.
\end{remark}

\begin{remark}
  Note that if $X_1, \dots, X_n \sim \Ber(1 / 2)$
  are independent, then we know that
  $S_n \sim \Bin(n, 1 / 2)$, which is not normal.
  So we have to interpret the central limit theorem
  in the appropriate sense.
\end{remark}

\begin{remark}
  Here are a few remarks about the central limit
  theorem:
  \begin{itemize}
    \item It is crucial in the central limit theorem
      that
      $\EE[X_k^2] < \infty$ for each $k$. One can even
      prove a converse in the sense that if the
      central limit theorem holds, then the
      second moments are finite.
    \item On the other hand, independence is
      not so crucial. The central limit theorem
      holds under a notion of ``weak dependence.''
      Note that uncorrelated is not enough, however.
    \item The order of convergence of the
      central limit theorem is
      $O(1 / \sqrt{n})$.
  \end{itemize}
\end{remark}

\section{Applications of the Central Limit Theorem}

\begin{example}
  Consider the following problem: How many
  times should a fair coin be tossed to be sure
  with probability at least $0.95$ that the
  proportion of
  tails is between $0.49$ and $0.51$.

  Note that flipping a fair coin $n$ times
  is like studying independent
  $X_1, \dots, X_n \sim \Bin(1 / 2)$. We want to
  find $n$ such that
  \[
    \PP\left(0.49 \le \frac{X_1 + \dots + X_n}{n} \le 0.51\right) \ge 0.95.
  \]
  One way to proceed is to note that
  $X_1 + \dots + X_n = \Ber(n, 1 / 2)$, so
  \[
    0.95 \le \PP(0.49n \le X_1 + \dots + X_n \le 0.51n)
    = \sum_{k = 0.49n}^{0.51n} \binom{n}{k} 2^{-n},
  \]
  i.e. we need to find $n$ such that
  \[
    \sum_{k = 0.49n}^{0.51n} \binom{n}{k} \ge 2^n(0.95).
  \]
  This is difficult to compute. On the other hand,
  the central limit theorem says that
  \[
    \mathcal{N}(0, 1)
    \approx \frac{(X_1 + \dots + X_n) - n / 2}{\sqrt{n / 4}}
    = \frac{2}{\sqrt{n}} \left(X_1 + \dots + X_n - \frac{n}{2}\right).
  \]
  Then if we let $Z_n = (X_1 + \dots + X_n - n / 2) / \sqrt{n / 4} \approx \mathcal{N}(0, 1)$, we have
  \begin{align*}
    \PP(0.49n \le X_1 + \dots + X_n \le 0.51n)
    &= \PP\left(-\frac{0.01n}{\sqrt{n / 4}} \le \frac{(X_1 + \dots + X_n) - n / 2}{\sqrt{n / 4}} \le \frac{0.01n}{\sqrt{n / 4}}\right) \\
    &\approx \PP(-0.02 \sqrt{n} \le Z_n \le 0.02 \sqrt{n})
  \end{align*}
  The above is $\Phi(0.02 \sqrt{n}) - \Phi(-0.02 \sqrt{n})$, so we would like
  \[
    0.95 \le \Phi(0.02 \sqrt{n}) - \Phi(-0.02 \sqrt{n})
    = 2\Phi(0.02 \sqrt{n}) - 1,
  \]
  or $\Phi(0.02 \sqrt{n}) \ge 0.975$. One can use
  a table or a calculator to find
  $\sqrt{n} \ge 98$, or $n \ge 98^2 \approx 10^4$.
\end{example}

\begin{example}
  Let $Y = X_1 + \dots + X_{15}$ be the sum of a
  random sample of size 15 (i.e. $X_1, \dots, X_{15}$
  are i.i.d. with a common distribution)
  from the continuous distribution with pdf given
  by
  \[
    f(x) = \frac{3}{2} x^2 \quad \text{for $-1 < x < 1$}.
  \]
  We would like to find an approximate
  value for
  $\PP(-0.3 \le Y \le 1.5)$. On one hand,
  $Y$ has pdf given by
  \[
    f_Y = f_{X_1} * \dots * f_{X_{15}},
  \]
  but this is very difficult to compute. Instead,
  we can use a normal approximation. We can compute
  \[
    \EE[Y] = 15 \cdot \EE[X_1]
    = 15 \int_{-1}^1 x \cdot \frac{3}{2} x^2 \, dx
    = 0
  \]
  since the integrand is odd. We can also compute
  the variance by
  \[
    \Var[Y] = 15 \cdot \Var[X_1]
    = 15 \int_{-1}^1 x^2 \cdot \frac{3}{2} x^2 \, dx
    = \frac{45}{2}\left[\frac{x^5}{5}\right]_{-1}^1
    = \frac{45}{2} \cdot \frac{2}{5} = 9.
  \]
  Thus $Y \approx \mathcal{N}(0, 9)$, i.e.
  $Y / 3 \approx \mathcal{N}(0, 1)$. Using this
  approximation, we obtain
  \[
    \PP(-0.3 \le Y \le 1.5)
    \approx \PP(-0.1 \le \mathcal{N}(0, 1) \le 0.5)
    = \Phi(0.5) - \Phi(-0.1) \approx 0.2313.
  \]
\end{example}

\begin{remark}
  As a general rule, when the distribution is
  not known (i.e. we must estimate the mean and
  variance from the samples), the central limit
  theorem can be used for $n \ge 30$. However, if
  we have knowledge of the population
  mean and variance, then $n \ge 10$ is usually
  sufficient.
\end{remark}

\section{The Poisson Approximation to the Binomial}
\begin{remark}
  We saw that if $X \sim \Bin(n, p)$
  with $n \gg 1$ and $p \ll 1$, then
  $X \approx Y$ where $Y \sim \Poi(\lambda = np)$.
\end{remark}

\begin{prop}
  Let $X_1, \dots, X_n$ be independent Poisson
  random variables with parameters
  $\lambda_1, \dots, \lambda_n$. Then we have
  $X_1 + \dots + X_n \sim \Poi(\lambda_1 + \dots + \lambda_n)$.
\end{prop}

\begin{proof}
  The moment generating function of
  $X \sim \Poi(\lambda)$ is
  \[
    M_X(t) = \EE[e^{tX}]
    = \sum_{k = 0}^\infty e^{tk} \frac{\lambda^k e^{-\lambda}}{k!}
    = e^{-\lambda }\sum_{k = 0}^\infty \frac{(\lambda e^t)^k}{k!}
    = e^{-\lambda} e^{\lambda e^t}
    = e^{\lambda(e^t - 1)}
  \]
  for all $t \in \R$. Then by independence,
  we have
  \[
    M_{X_1 + \dots + X_n}(t)
    = \prod_{k = 1}^n M_{X_k}(t)
    = \prod_{k = 1}^n \exp(\lambda_k(e^t - 1))
    = \exp((\lambda_1 + \dots + \lambda_n)(e^t - 1)).
  \]
  Thus by uniqueness, we have
  $X_1 + \dots + X_n \sim \Poi(\lambda_1 + \dots + \lambda_n)$.
\end{proof}

\begin{remark}
  Thus for sums of independent Bernoulli random
  variables, we have two types of approximations:
  \begin{itemize}
    \item Poisson: $X_1 + \dots + X_n \approx \Poi(np)$.
    \item Normal: $X_1 + \dots + X_n \approx \mathcal{N}(np, np(1 - p))$.
  \end{itemize}
  The two approximations are of the same order
  because $\Poi(\lambda) \approx \mathcal{N}(\lambda, \lambda)$.
  Thus if $p \ll 1$, then
  \[
    \Poi(np) \approx \mathcal{N}(np, np)
    \approx \mathcal{N}(np, np(1 - p)).
  \]
  The approximation $\Poi(\lambda) \approx \mathcal{N}(\lambda, \lambda)$
  is to be understood via their distribution
  functions, i.e. that
  \[
    \PP(Y \le y)
    \approx \PP(X \le y)
  \]
  if we have $Y \sim \Poi(\lambda)$ and
  $X \sim \mathcal{N}(\lambda, \lambda)$.
\end{remark}

\section{Homework Problems}
Problems \#1, 2, 5, 7, 8, 11, 14, 15
from Grimmett and Welsh.
