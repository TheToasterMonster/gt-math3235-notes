\documentclass[12pt, letterpaper, oneside]{book}
\usepackage[margin={0.6in, 0.75in}]{geometry}
\usepackage{microtype}
% \usepackage{kpfonts}
\usepackage{amsmath, amssymb, amsthm}
\usepackage{parskip}
\usepackage[many]{tcolorbox}
\usepackage{footnote}
\usepackage{cancel}
\usepackage{titlesec}
\usepackage{pgffor}
\usepackage[shortlabels]{enumitem}
\usepackage{hyperref}
\usepackage{tikz-cd}

\usepackage[overload]{textcase}

\newtheorem{axiom}{Axiom}[chapter]
\newtheorem{theorem}{Theorem}[chapter]
\newtheorem{prop}{Proposition}[chapter]
\newtheorem{corollary}{Corollary}[theorem]
\newtheorem{lemma}{Lemma}[chapter]
\theoremstyle{definition}
\newtheorem{definition}{Definition}[chapter]
\newtheorem{exercise}{Exercise}[chapter]
\newtheorem{example}{Example}[definition]
\newtheorem*{remark}{Remark}

\tcbset{sharp corners, breakable, enhanced, parbox=false}
\newtcolorbox{mybox}[3][]
{
  colframe = #2!150,
  colback  = #2!5,
  coltitle = #2!0!white,  
  title    = {#3},
  #1,
}

\titleformat{\chapter}[display]
    {\normalfont\huge\bfseries}{\chaptertitlename\ \thechapter}{20pt}{\Huge}
\titlespacing*{\chapter}{0pt}{0pt}{40pt}

\newcommand{\R}{\mathbb{R}}
\newcommand{\N}{\mathbb{N}}
\newcommand{\Z}{\mathbb{Z}}
\newcommand{\C}{\mathbb{C}}
\newcommand{\Q}{\mathbb{Q}}
\newcommand{\F}{\mathcal{F}}
\newcommand{\PP}{\mathbb{P}}
\newcommand{\EE}{\mathbb{E}}

\DeclareMathOperator{\Vol}{Vol}
\DeclareMathOperator{\Int}{int}
\DeclareMathOperator{\area}{area}
\DeclareMathOperator{\curl}{curl}
\DeclareMathOperator{\card}{card}
\DeclareMathOperator{\Var}{Var}

\def\indep{\perp\!\!\!\perp}

\title{MATH 3235: Probability Theory}
\author{Frank Qiang\\Instructor: Christian Houdre}
\date{Georgia Institute of Technology\\Fall 2024}

\begin{document}
  \maketitle

  \begingroup
  \let\cleardoublepage\clearpage
  \tableofcontents
  \endgroup

  % \foreach \i in {00, 01, 02, 03, 04, ..., 50} {%
  %   \edef\FileName{lectures/lecture\i.tex}%     The % here are necessary to eliminate any
  %   \IfFileExists{\FileName}{%  spurious spaces that may get inserted
  %      \input{\FileName}%       at these points
  %   }
  % }
  \chapter{Events and Probabilities}

\section{Probability Spaces}

\begin{definition}
  A \emph{probability space} is a triple
  $(\Omega, \mathcal{F}, \mathbb{P})$
  where
  \begin{itemize}
    \item $\Omega$ is called the \emph{sample space}
      (the set of all possible
      outcomes of a random experiment);
    \item $\mathcal{F} \subseteq \mathcal{P}(\Omega)$,
      called the \emph{event space},\footnote{The elements of $\mathcal{F}$ are called \emph{events}. Events with cardinality 1 are called \emph{elementary}.} is nonempty
      and must satisfy:
      \begin{enumerate}[(i)]
        \item if $A \in \mathcal{F}$, then
          $A^c \in \mathcal{F}$,
        \item if $A_1, A_2, \dots \in \mathcal{F}$,
          then
          $\bigcup_{i=1}^\infty A_i \in \mathcal{F}$;
      \end{enumerate}
    \item $\mathbb{P}$ is a probabilty measure
      on $(\Omega, \mathcal{F})$ (to be defined later).
  \end{itemize}
\end{definition}

\begin{remark}
  In general, when $\Omega$ is finite or countably
  infinite, one takes
  $\mathcal{F} = \mathcal{P}(\Omega)$.
\end{remark}

\begin{prop}
  We always have $\varnothing, \Omega \in \mathcal{F}$.
\end{prop}

\begin{proof}
  Since $\mathcal{F} \ne \varnothing$, there
  exists some event $A \in \mathcal{F}$. Then we get $A^c \in \mathcal{F}$
  and $\Omega = A \cup A^c \in \mathcal{F}$
  by the complement and union properties of
  $\mathcal{F}$. Finally
  $\varnothing = \Omega^c \in \mathcal{F}$
  by the complement property.
\end{proof}

\begin{definition}
  A \emph{probability measure} on $(\Omega, \mathcal{F})$ is a function
  $\mathbb{P} : \mathcal{F} \to [0, \infty)$
  such that
  \begin{enumerate}[(i)]
    \item $\mathbb{P}(\Omega) = 1$,
    \item and $\mathbb{P}(\bigcup_{k = 1}^\infty A_k) = \sum_{k = 1}^\infty \mathbb{P}(A_k)$
      whenever $A_1, A_2, \dots \in \mathcal{F}$ are pairwise disjoint.\footnote{i.e. $A_i \cap A_j = \varnothing$ whenever $i \ne j$.}
  \end{enumerate}
\end{definition}

\begin{prop}
  The following properties hold for any
  probability measure $\mathbb{P}$ on
  $(\Omega, \mathcal{F})$:
  \begin{enumerate}[(1)]
    \item For any $A \in \mathcal{F}$,
      we have $\mathbb{P}(A^c) = 1 - \mathbb{P}(A)$.
    \item Let $A, B \in \mathcal{F}$ with
      $A \subseteq B$. Then $\mathbb{P}(A) \le \mathbb{P}(B)$.
    \item Let $A, B, C \in \mathcal{F}$. Then
      \[
        \PP(A \cup B) = \PP(A) + \PP(B) - \PP(A \cap B).
      \]
      This is the \emph{principle of inclusion-exclusion}.
  \end{enumerate}
\end{prop}

\begin{proof}
  $(1)$ Observe that $A \cup A^c = \Omega$ and
  $A \cap A^c = \varnothing$, so
  $1 = \mathbb{P}(\Omega) = \mathbb{P}(A \cup A^c) = \mathbb{P}(A) + \mathbb{P}(A^c)$.

  $(2)$ Write $B = A \cup (B \setminus A)$.\footnote{Note that $B \setminus A \in \mathcal{F}$ since $B \setminus A = B \cap A^c = (B^c \cup A)^c$.} Since
  $A \cap (B \setminus A) = \varnothing$,
  we have $\mathbb{P}(B) = \mathbb{P}(A) + \mathbb{P}(B \setminus A) \ge \mathbb{P}(A)$.\footnote{Since $\mathbb{P} : \mathcal{F} \to [0, \infty)$, we have $\mathbb{P}(B \setminus A) \ge 0$.}

  $(3)$ Left as an exercise. Follow similar ideas
  as in (2).
\end{proof}

\begin{remark}
  Observe that property (2) implies
  $\mathbb{P}(A) \le \mathbb{P}(\Omega) = 1$
  since any $A \subseteq \Omega$.
\end{remark}

\begin{example}
  Pick a point uniformly at random from the
  unit square $\Omega = [0, 1] \times [0, 1]$
  and record its coordinates. Then the probability 
  of the point being inside a fixed shape
  $S \subseteq \Omega$ is $|S|$, the area of $S$.
\end{example}

\begin{remark}
  Note that $\PP$ only satisfies
  \emph{countable} additivity. For instance let
  $\Omega = [0, 1]$ and $\PP$ be the uniform
  measure on $\Omega$. Then $\Omega = \bigcup_{x \in [0, 1]} \{x\}$
  and $\PP(\{x\}) = 0$ for every $x \in [0, 1]$,
    but $\PP(\Omega) = 1$.
    This is because the union
    $\bigcup_{x \in [0, 1]} \{x\}$ is uncountable.
\end{remark}

\begin{definition}
  Let $\Omega$ be finite and
  $\mathcal{F} = \mathcal{P}(\Omega)$. The uniform
  probability on $(\Omega, \mathcal{F})$
  is the one such that
  \[
    \mathbb{P}(\{\omega\}) = \frac{1}{\card \Omega}
    \quad \text{for all $\omega \in \Omega$.}
  \]
\end{definition}

\begin{prop}
  Let $\PP$ be the uniform probability
  on a finite set $\Omega$ and let $A \in \mathcal{F}$.
  Then
  \[
    \PP(A) = \frac{\card A}{\card \Omega}.
  \]
\end{prop}

\begin{proof}
  Note that $A$ is finite since $\Omega$ is and
  so we may enumerate its elements
  as $A = \{\omega_1, \omega_2, \dots, \omega_n\}$,
  where $n = \card A$.
  Then the sets $\{\omega_i\}_{i = 1}^n$ are pairwise disjoint
  and thus we have
  \[
    \PP(A) = \PP\left(\bigcup_{i = 1}^n \{\omega_i\}\right)
    = \sum_{i = 1}^n \PP(\{\omega_i\})
    = \sum_{i = 1}^n \frac{1}{\card \Omega}
    = \frac{n}{\card \Omega}
    = \frac{\card A}{\card \Omega},
  \]
  which is the desired result.
\end{proof}

\section{Conditional Probability}
\begin{definition}
  Let $B \in \mathcal{F}$ such that
  $\PP(B) > 0$. Then the \emph{conditional probability}
  of $A$ given $B$, written $\PP(A | B)$, is given by
  \[
    \PP(A | B) = \frac{\PP(A \cap B)}{\PP(B)}.
  \]
\end{definition}

\begin{remark}
  The intuition is that the extra information gained by
  knowing the occurrence of $B$ should update our
  computation of the probability of $A$.
\end{remark}

\begin{remark}
  Another way to think about conditional probability
  is a restriction of the sample space to $B$.
\end{remark}

\section{Homework Problems}

Problems \#1, 2, 9, 10, 14 from Grimmett and Welsh.

  \chapter{Discrete Random Variables}

\section{Probability Mass Functions}
\begin{example}
  Consider the following game: Flip a fair
  coin 10 times and roll a fair die. I give you
  \[
    (\text{number of heads}) \times (\text{number on die})
    \text{ dollars}.
  \]
  This is a simple game, but it is kind of
  painful to write in terms of events
  (e.g. $\PP(\text{win} \ge \$10))$).
  We would have to set
  \[
    \Omega = \{\text{all sequences like } (H, T, H, H, T, T, T, T, T, H, 4)\}
  \]
  and $\mathcal{F} = \mathcal{P}(\Omega)$. It is
  also not immediately obviously which
  sequences are in $\{\text{win} \ge \$10\}$. Instead,
  we would prefer something like
  \begin{quote}
    ``Let $H$ be the number of heads in 10
    fair coin tosses and let $R$ be the outcome
    of a roll of a fair die. Then you get $HR$ dollars.''
  \end{quote}
  How do we do this in our axiomatic framework?
  What are $H, R$? Here are some observations:
  \begin{itemize}
    \item $H, R$ are real numbers,
    \item and they are determined by the outcome of the
      experiment.
  \end{itemize}
  Thus we should think of $H, R$ as functions
  from $\Omega$ to $\R$. These are examples of
  \emph{discrete random variables}.
\end{example}

\begin{remark}
  The name ``random variable'' is just
  historic. Really, $H, R$ are non-random functions.
\end{remark}

\begin{remark}
  Can every function $X : \Omega \to \R$
  be a discrete random variable? Note that we want to
  talk about probabilities like
  $\PP(X = 17)$. This indicates that the event
  \[
    \{X = 17\} = \{\omega \in \Omega : X(\omega) = 17\}
  \]
  has to be in $\mathcal{F}$. So we require that
  $X$ is \emph{measurable}, i.e. for
  every $x \in \R$, we have
  $\{x \in \Omega : X(\omega) = x\} \in \mathcal{F}$.
  Also $H, R$ must have special properties,
  for instance they can only take on finitely many
  values.
\end{remark}

\begin{definition}
  A function $X : \Omega \to \R$ is a
  \emph{discrete random variable} if
  \begin{enumerate}[(i)]
    \item for every $x \in \R$, we have
      $\{X = x\} \in \mathcal{F}$,
    \item and $X(\Omega) = \{x \in \R : x = X(\omega) \text{ for some } \omega\}$
      is finite or countably infinite.
  \end{enumerate}
\end{definition}

\begin{remark}
  Often, we only care about what values $X$
  can take and with what probabilities. We store this
  data in a special function called the
  \emph{probability mass function}.
\end{remark}

\begin{definition}
  Let $X$ be a discrete random variable. Then
  its \emph{probability mass function (pmf)} is
  \[
    p_X : \R \to [0, 1] \quad \text{defined by} \quad p_X(s) = \PP(X = s).
  \]
\end{definition}

\begin{example}
  Let $X$ be the outcome of the roll of a fair die.
  Then
  \[
    p_X(s) = \begin{cases}
      1/6 & \text{if } s \in \{1, 2, 3, 4, 5, 6\}, \\
      0 & \text{otherwise}.
    \end{cases}
  \]
\end{example}

\begin{remark}
  Another sentence we want to say is:
  \begin{quote}
    ``A discrete random variable $X$ takes
    values $\{1, 7, 9\}$ with probabilities
    $1 / 2, 1 / 3, 1 / 6$, respectively if and only
    if
    \[
      p_X(s) =
      \begin{cases}
        1 / 2 & \text{if } s = 1, \\
        1 / 3 & \text{if } s = 7, \\
        1 / 6 & \text{if } s = 9, \\
        0 & \text{otherwise}.
      \end{cases}
    \]
  \end{quote}
  How do we know this exists? In other words,
  does there exist
  $(\Omega, \mathcal{F}, \PP)$ and
  $X : \Omega \to \R$ with this pmf?
\end{remark}

\begin{theorem}
  Let $S = \{s_i : i \in I\}$ be a countable
  subset of $\R$ and let
  $\{\pi_i : i \in I\}$ be a collection of
  numbers such that $\pi_i \ge 0$ and
  \[
    \sum_{i \in I} \pi_i = 1.
  \]
  Then there exists a probability space
  $(\Omega, \mathcal{F}, \PP)$ and a discrete
  random variable $X : \Omega \to \R$ such that
  \[
    p_X(s) =
    \begin{cases}
      \pi_i & \text{if } s = s_i, \\
      0 & \text{otherwise}.
    \end{cases}
  \]
\end{theorem}

\begin{proof}
  Take $\Omega = S$ and
  $\mathcal{F} = \mathcal{P}(S)$. Set
  \[\PP(A) = \sum_{i : s_i \in A} \pi_i\]
  and define $X : \Omega \to \R$ given by
  $X(\omega) = \omega$. Then one can check
  that $X$ has the desired pmf.
\end{proof}

\begin{remark}
  This allows us to just say
  \begin{quote}
    ``Let $X$ be a discrete random variable taking
    these values with these probabilities''
  \end{quote}
  without worrying about the underlying
  $(\Omega, \mathcal{F}, \PP)$.
\end{remark}

\section{Common Discrete Random Variables}
\begin{example}
  Some common examples of discrete random variables
  are:
  \begin{enumerate}
    \item \emph{Constant random variables}:
      Define $X : \Omega \to \R$ by
      $\omega \mapsto X(\omega) = C$.
    \item \emph{Bernoulli random variables}:
      For $0 < p < 1$, we say that
      $X \sim \mathrm{Ber}(p)$ if
      \[
        X =
        \begin{cases}
          1 & \text{with probability } p, \\
          0 & \text{with probability } q = 1 - p.
        \end{cases}
      \]
      This models a possibly unfair
      coin flip. The Bernoulli random variable $X$
      has pmf
      \[
        p_X(s) =
        \begin{cases}
          p & \text{if } s = 1, \\
          1 - p & \text{if } s = 0, \\
          0 & \text{otherwise}.
        \end{cases}
      \]
    \item \emph{Binomial random variables}:
      For $n \in \N^* = \N \setminus \{0\}$ and $0 < p < 1$, we say that
      $X \sim \mathrm{Bin}(n, p)$ if
      \[
        \PP(X = k) = \binom{n}{k} p^k (1 - p)^{n - k}
      \]
      for $k = 0, 1, \dots, n$ and $\PP(X = k) = 0$
      otherwise. To that this is indeed a pmf,
      observe that
      \[
        \sum_{k = 0}^n \PP(X = k) = \sum_{k = 0}^n \binom{n}{k} p^k (1 - p)^{n - k} = (p + (1 - p))^n = 1^n = 1.
      \]
      The $n = 1$ case reduces to a Bernoulli
      random variable.
    \item \emph{Geometric random variables}:
      For $0 < p < 1$, we say that
      $X \sim \mathrm{Geo}(p)$ if
      \[
        \PP(X = k) = p(1 - p)^{k - 1}
      \]
      for $k = 1, 2, 3, \dots$ and
      $\PP(X = k) = 0$ otherwise. The above function
      is clearly nonnegative and
      \[
        \sum_{k = 1}^\infty p(1 - p)^{k - 1}
        = \frac{p}{1 - (1 - p)} = 1,
      \]
      so this is indeed a pmf. The geometric
      random variable models the
      number of independent Bernoulli trials needed
      to obtain the first success.
  \end{enumerate}
\end{example}

\begin{example}
  Consider the random variable $X$ which counts
  the number of independent Bernoulli trials
  needed to get the 4th success. Note that the
  range of $X$ is $\{4, 5, 6, \dots\}$. Then
  \[
    \PP(X = k) = \binom{k - 1}{3} p^3(1 - p)^{k - 4} p
    = \binom{k - 1}{3} p^4 (1 - p)^{k - 4}
  \]
  for $k = 4, 5, 6, \dots$ and $\PP(X = k) = 0$
  otherwise.
  This is because
  the last trial must be a success and the
  previous $k - 1$ trials need to contain $3$
  successes. Here $X = \mathrm{NBin}(n = 4, p)$,
  the \emph{negative binomial random variable}. In general,
  $X \sim \mathrm{NBin}(n, p)$ takes on values
  $n, n + 1, n + 2, \dots$ and
  \[
    \PP(X = k) = \binom{k - 1}{n - 1} p^n (1 - p)^{k - n}
  \]
  for $k = n, n + 1, n + 2, \dots$. Note that
  the $n = 1$
  case reduces to a geometric random variable.
  The name
  comes from the binomial theorem with negative
  exponents.
\end{example}

\begin{example}
  We say that $X$ is a \emph{Poisson random variable}
  with parameter $\lambda > 0$, written
  $X \sim \mathrm{Poi}(\lambda)$, if $X$ takes the
  values $k = 0, 1, 2, \dots$ with probability mass
  function
  \[
    p_X(k) = \PP(X = k) = \frac{e^{-\lambda} \lambda^k}{k!}.
  \]
  Note that $p_X$ is clearly nonnegative and
  \[
    \sum_{k = 0}^\infty p_X(k) = \sum_{k = 0}^\infty \frac{e^{-\lambda} \lambda^k}{k!}
    = e^{-\lambda} \sum_{k = 0}^\infty \frac{\lambda^k}{k!}
    = e^{-\lambda} e^\lambda = 1,
  \]
  so $p_X$ is indeed a pmf. One can view the
  Poisson random variable in the following manner:
  Suppose $X \sim \mathrm{Bin}(n, p)$ with
  $n \gg 1$ and $p \ll 1$, e.g. $n = 10^5$ and
  $p = 10^{-4}$. Then
  \[
    \PP(X = 100) =
    \binom{10^5}{100} \left(\frac{1}{10^4}\right)^{100}
    \left(1 - \frac{1}{10^4}\right)^{10^5 - 100}.
  \]
  This is very difficult to compute.
  Instead, we approximate this via the Poisson
  random variable.
\end{example}

\begin{prop}
  Let $n \to \infty$ and $p = p(n) \to 0$ in such
  a way that $np(n) \to \lambda > 0$ as $n \to \infty$.
  Then
  \[
    \binom{n}{k} p^k (1 - p)^{n - k} \xrightarrow[n \to \infty]{} \frac{e^{-\lambda} \lambda^k}{k!},
  \]
  i.e. $p_X(k) \to p_Y(k)$ pointwise for
  $k = 0, 1, 2, \dots$, where $X \sim \mathrm{Bin}(n, p)$
  and $Y \sim \mathrm{Poi}(\lambda)$.
\end{prop}

\begin{proof}
  Observe that
  \begin{align*}
    \binom{n}{k} p^k (1 - p)^{n - k}
    &= \frac{n!}{k!(n - k)!} p^k (1 - p)^{n - k}
    = \frac{1}{k!} \left[n(n - 1) \dots (n - k + 1)p^k(1 - p)^{-k} (1 - p)^n\right] \\
    &= \frac{1}{k!} \left[\frac{n(n - 1) \dots (n - k + 1)}{n^k} n^k p^k (1 - p)^{-k} (1 - p)^n\right].
  \end{align*}
  Now notice that $n^k p^k = (np)^k \to \lambda^k$
  since $np \to \lambda$, and
  \[
    \lim_{n \to \infty} \frac{n(n - 1) \dots (n - k + 1)}{n^k} = 1
    \quad \text{and} \quad
    \lim_{p \to 0} (1 - p)^{-k} = 1.
  \]
  Finally, setting $\lambda = np$,
  \[
    (1 - p)^n =
    \left(1 - \frac{\lambda}{n}\right)^n
    \to e^{-\lambda}.
  \]
  Putting all of this together, we see that
  \[
    \binom{n}{k} p^k (1 - p)^{n - k} \xrightarrow[n \to \infty]{} \frac{e^{-\lambda} \lambda^k}{k!},
  \]
  which is the desired result.
\end{proof}

\section{Expectation of Random Variables}
\begin{remark}
  Suppose $X : \Omega \to \R$ is a discrete random variable and
  $h : \R \to \R$. Then we have:
  \[
    \begin{tikzcd}
      \Omega \ar[r, "X"] \ar[dr, "h(X)", swap] & \R \ar[d, "h"] \\
      & \R
    \end{tikzcd}
  \]
  In particular, $h \circ X : \Omega \to \R$ is
  also a random variable.
\end{remark}

\begin{definition}
  Let $X$ be a discrete random variable. The
  \emph{(mathematical) expectation} of $X$ is\footnote{We write $\mathcal{R}(X)$ to denote the range of $X$.}
  \[
    \EE[X] = \sum_{x \in \mathcal{R}(X)} x p_X(x)
  \]
  if the above sum exists and converges
  absolutely,\footnote{i.e. $\sum_{x \in \mathcal{R}(X)} |x| p_X(x) < \infty$.}
  where
  $p_X$ is the probability mass function of $X$.
\end{definition}

\begin{remark}
  When $X$ is discrete, the expectation coincides
  with the usual notion of a mean. In general,
  the expectation is some kind of weighted mean.
\end{remark}

\begin{remark}
  Observe that the sum in the definition of
  $\EE[X]$ need not converge. Even worse, if it
  only converges conditionally, then by the Riemann
  rearrangement theorem we may get any real value
  we wish by reordering the sum. This is why we
  require absolute convergence.
\end{remark}

\begin{example}
  Set $Y = X^2$. Then we have
  \[
    \EE[X^2] = \EE[Y] = \sum_{y \in \mathcal{R}(Y)} y \PP(Y = y)
    = \sum_{y \in \mathcal{R}(X^2)} y \PP(X^2 = y).
  \]
  If we explicitly let
  \[
    X =
    \begin{cases}
      1 & \text{with probability } 1 / 2, \\
      -1 & \text{with probability } 1 / 2,
    \end{cases}
  \]
  we see that $\EE[X] = 0$. We can also see that
  $\EE[X^2] = 1$ since $X^2 = 1$ with probability $1$.
  Equivalently, we can compute that
  \[
    \EE[X^2]
    = \sum_{y \in \mathcal{R}(X^2)} y \PP(X^2 = y)
    = 1 \cdot \PP(X^2 = 1) = 1.
  \]
\end{example}

\begin{prop}[Law of the unconcious statistician]
  For any $h : \R \to \R$ and $X : \Omega \to \R$
  discrete,
  \[
    \EE[h(X)] = \sum_{x \in \mathcal{R}(X)} h(x) p_X(x)
  \]
  where $p_X$ is the pmf of $X$, provided
  these sums exist and converge absolutely.
\end{prop}

\begin{proof}
  Let $Y = h(X)$. Then we have
  \[
    \EE[h(X)] = \EE[Y]
    = \sum_{y \in \mathcal{R}(Y)} y p_Y(y)
    = \sum_{y \in \mathcal{R}(Y)} y \PP(h(X) = y)
    = \sum_{x \in \mathcal{R}(X)} h(x) \PP(h(X) = y).
  \]
  Note that that $y$ in the last term is
  $h(x)$, and thus $\PP(h(X) = y) = \PP(h(X) = h(x)) = p_X(x)$.
  Then
  \[
    \EE[h(X)] = \sum_{x \in \mathcal{R}(X)} h(x) p_X(x),
  \]
  which is precisely the desired result.
\end{proof}

\begin{remark}
  In the discrete case, we do not require
  that $h$ be measurable
  since $\mathcal{R}(X)$ is at most countable.
\end{remark}

\begin{prop}
  We have the following properties of expectation:
  \begin{enumerate}[(i)]
    \item If $X \ge 0$, then $\EE[X] \ge 0$.
    \item If $X = C$ is constant, then
      $\EE[X] = C$.
    \item If $\EE[aX + bY] = a\EE[X] + b\EE[Y]$.
  \end{enumerate}
\end{prop}

\begin{proof}
  $(i)$ Since $X \ge 0$, we have $\mathcal{R} \subseteq [0, \infty)$
  and thus
  \[
    \EE[X] = \sum_{x \in \mathcal{R}(X)} x p_X(x) \ge 0
  \]
  since every term in the sum is nonnegative.

  $(ii)$ Since $\mathcal{R}(X) = \{C\}$, we have
  $\PP(X = C) = 1$ and thus
  \[
    \EE[X] = \sum_{x \in \mathcal{R}(X)} x p_X(x)
    = C \cdot \PP(X = C) = C.
  \]
  This is the desired result.

  $(iii)$ We compute that
  \begin{align*}
    \EE[aX + bY]
    &= \sum_{ax + by \in \mathcal{R}(aX + bY)} (ax + by) p_{aX + bY}(ax + by)
    = \sum_{x \in \mathcal{R}(X)} ax p_{X}(x)
    + \sum_{y \in \mathcal{R}(Y)} by p_{Y}(y) \\
    &= a \sum_{x \in \mathcal{R}(X)} x p_{X}(x)
    + b \sum_{y \in \mathcal{R}(Y)} y p_{Y}(y)
    = a\EE[X] + b\EE[Y],
  \end{align*}
  which is the desired equality.
\end{proof}

\begin{example} We compute the following:
  \begin{enumerate}
    \item Let $X \sim \mathrm{Ber}(p)$. Then
      $\EE[X] = 0(1 - p) + 1p = p$.
    \item Let $X \sim \mathrm{Bin}(n, p)$. Then
      \begin{align*}
        \EE[X]
        &= \sum_{k = 0}^n k \binom{n}{k} p^k (1 - p)^{n - k}
        = \sum_{k = 1}^n k \binom{n}{k} p^k (1 - p)^{n - k}
        = \sum_{k = 1}^n k \frac{n!}{k!(n - k)!} p^k (1 - p)^{n - k} \\
        &= \sum_{k = 1}^n \frac{n!}{(k - 1)!(n - k)!} p^k (1 - p)^{n - k}
        = n \sum_{k = 1}^n \frac{(n - 1)!}{(k - 1)!(n - k)!} p^k (1 - p)^{n - k} \\
        &= n \sum_{k = 1}^n \binom{n - 1}{k - 1} p^k (1 - p)^{n - k}
        = np \sum_{k = 1}^n \binom{n - 1}{k - 1} p^{k - 1} (1 - p)^{n - 1 - (k - 1)} = np.
      \end{align*}
      In the last step we re-index with $j = k - 1$,
      and then recognize the terms as the pmf
      of a $\mathrm{Bin}(n - 1, p)$ random variable,
      which must sum to $1$ over $0 \le j \le n - 1$.
    \item Let $X \sim \mathrm{Geo}(p)$. Then
      \begin{align*}
        \EE[X]
        &= \sum_{k = 1}^\infty k p(1 - p)^{k - 1}
        = p \sum_{k = 1}^\infty k(1 - p)^{k - 1}
        = p \sum_{k = 1}^\infty \frac{d}{dx}\left. (1 - x)^k\right|_{x = p}
        = p \frac{d}{dx} \left.\sum_{k = 1}^\infty (1 - x)^k\right|_{x = p} \\
        &= p \frac{d}{dx} \left.\frac{1 - x}{1 - (1 - x)}\right|_{x = p}
        = p \frac{d}{dx} \left.\frac{1 - x}{x}\right|_{x = p}
        = p \frac{d}{dx} \left.\left(1 - \frac{1}{x}\right)\right|_{x = p}
        = p \cdot \frac{1}{p^2} = p.
      \end{align*}
      The exchange of the sum and derivative is
      justified since $0 < p < 1$, so we are in the
      region of uniform convergence of the power
      series.
  \end{enumerate}
\end{example}

\section{Homework Problems}
Problems \#1, 2, 4, 5, 6, 7, 9, 10 from Grimmett and Welsh.

\end{document}
